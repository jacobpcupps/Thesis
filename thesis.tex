\documentclass[12pt,letterpaper]{article}
\usepackage[utf8]{inputenc}
\usepackage[T1]{fontenc}
\usepackage{amsmath}
\usepackage{amsfonts}
\usepackage{amssymb}
\usepackage{graphicx}
\usepackage{setspace}
\usepackage{palatino}
\usepackage[notes,backend=biber]{biblatex-chicago}
\bibliography{Sesame.bib}
\usepackage[left=1.20in, right=1.20in, top=1.20in, bottom=1.20in]{geometry}
\begin{document}
	
	\title{The Classical Musical Educiational Project of \textit{Sesame Street}}
	\author{Eden Attar}
	\maketitle
	
	\doublespacing
	\thispagestyle{empty}

	\noindent
	On a summer evening in Saint Louis I was listening to the radio in
	the kitchen. In the research for this paper, which had grown from a character
	study on Yo-Yo Ma to encompass the themes of eduction and public broadcasting,
	I had decided that I should actually own some kind of broadcast receiver. 
	After looking into mini televisions, I settled on the conservative option
	of a radio. As I chopped onions, my local NPR station's weekend programming
	buzzed in the background.
	
	A program on the recent events in Afghanistan had ended and the PSA 
	space, where a commercial station
	would have run adds, began. I heard a woman's voice say ``STLPR: understanding
	starts here.'' 

	My ears perked up. What an intriguing declaration. It is not ``STLPR:
	your source of news,'' or ``STLPR: hear the nation.'' Instead, the 
	slogan is a 
	promise to increase your \textit{understanding. }NPR is offering not
	just to 
	inform you, but moreover to provide some kind of ethical education,
	presumably into cultural-socio-political events such that one comes away
	*understanding*. 

	A strange word, ``understanding'' In the context of the socio-political,
	cultural and economic topics under the purview of national public 
	radio, something seems very kind about the word. It feels emotionally
	laden and steeped in empathy. 

	Its usage does not imply factual understanding, something like ``Let
	us factually explain what's going on in Afghanistan,'' but rather, 
	``cultivate an understanding of the human experiences of Afghanistan 
	along with us.'' 


Etymologically, the first half of the word understand does not mean the more modern "below" but rather, among or in between. To understand is to be among, and in a certain way to stand with. What NPR is offereing, then, is empathy in a broadcast. [https://www.etymonline.com/word/understand](https://www.etymonline.com/word/understand)


	This fits with the [z 995]founding rhetoric of NPR and its focus on hearing common people speak.[/z]
	

%\singlespacing
\printbibliography
\end{document}
